\title{Lottery-based decentralized verification for}
\author{Marcin Mielniczuk}
\documentclass[12pt]{article}

\usepackage{amsfonts}

\begin{document}
\maketitle

\begin{abstract}
We first present a distributed blockchain-based algorithm for computation verification. 
First, we'll present the general idea, ignoring the costs of blockchain message insertion. 
Later, we'll show how to gather the results in an efficient way.
\end{abstract}

\section{Assumptions}

We assume that the task is easily distributable, 
i.e. that it can be easily divided into multiple subtasks, 
each of them to be computed separately.

Additionally, assume the following:
\begin{enumerate}
	\item The provider outputs the percentage $c$ of correct results (e.g. random garbage).
	\item The percentage of good (honest) nodes in the network equals $p$, $p > \frac{1}{2}$.
	\item The whole task is constituted of $k$ subtasks
\end{enumerate}

\section{Algorithm}
Choose $N$ nodes from the network at random and assign exactly $n$ subtasks to each.
The task of each node is to recalculate all of the subtasks. Gathering the results 
consists of two phases

\begin{enumerate}
	\item Each node publishes an encrypted and signed verdict about the correctness of all $n$ subtasks (e.g. $1$ when all $n$ subtasks were correct, $0$ when there was any incorrect subtask)
	\item Each node publishes the decryption key for his own message
\end{enumerate}

After all nodes have published their decryption keys (or a timeout was reached), the results are gathered and the final verdict announced (or just calculated).
Precisely, we assume that the true verdict is the one the majority of peers has voted for.

This algorithm has an inherent risk of peers deciding not to reveal the decryption keys at some point in time. 
This will not be addressed since the blockchain-fee-optimal method doesn't suffer from this problem.

Notice that this approach gives us a great flexibility in the terms of computational redundancy. This is better than verification by replication whenever $N * n < k$.

\section{Probabilistic analysis}
Let $X$ be a random variable outputting the number of nodes who voted that the computation was carried out correctly.
Let's consider two scenarios. 
\subsection{Bad provider trying to get away with an incorrect computation}
With sufficiently large number of verifying nodes $N$, we can expect approximately $pN$ honest and $(1-p)N$ dishonest nodes. 
Assume the worst case: all dishonest peers are colluding. This means that 
\begin{equation}
\frac{\mathbb{E}X}{N} = (1-p) + pc^n
\end{equation}
with $\lim_{n\to\infty} c^n = 0$ it's just the matter of adjusting the parameter $n$ so that the dishonest peers cannot overthrow the verdict being in minority.

\subsection{Good provider penalized by colluding dishonest peers}
In this case all $pN$ peers provide the truthful verdict, so the requirement $p > \frac{1}{2}$ is enough.

\section{Game-theory analysis} 
TODO (why it pays to really compute it)

\section{Cheap verdict collection}
TODO
\section{Presentation on concrete parameter values}
TODO

\end{document}
