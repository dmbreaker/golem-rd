\title{Lottery-based decentralized verification for}
\author{Marcin Mielniczuk}
\documentclass[12pt]{article}

\usepackage{amsfonts}

\begin{document}
\maketitle

\begin{abstract}
We first present a distributed blockchain-based algorithm for computation verification. 
First, we'll present the general idea, ignoring the costs of blockchain message insertion. 
Later, we'll show how to gather the results in an efficient way.
\end{abstract}

\section{Assumptions}

We assume that the task is easily distributable, 
i.e. that it can be easily divided into multiple subtasks, 
each of them to be computed separately.

Additionally, assume the following:
\begin{enumerate}
	\item The provider outputs the percentage $c$ of correct results (e.g. random garbage).
	\item The percentage of good (honest) nodes in the network equals $p$, $p > \frac{1}{2}$.
	\item The whole task is constituted of $k$ subtasks
\end{enumerate}

\section{Algorithm}
Choose $N$ nodes from the network at random and assign exactly $n$ subtasks to each.
The task of each node is to recalculate all of the subtasks. Gathering the results 
consists of two phases

\begin{enumerate}
	\item Each node publishes an encrypted and signed verdict about the correctness of all $n$ subtasks (e.g. $1$ when all $n$ subtasks were correct, $0$ when there was any incorrect subtask)
	\item Each node publishes the decryption key for his own message
\end{enumerate}

After all nodes have published their decryption keys (or a timeout was reached), the results are gathered and the final verdict announced (or just calculated).
Precisely, we assume that the true verdict is the one the majority of peers has voted for.

This algorithm has an inherent risk of peers deciding not to reveal the decryption keys at some point in time. 
This will not be addressed since the blockchain-fee-optimal method doesn't suffer from this problem.

Notice that this approach gives us a great flexibility in the terms of computational redundancy. This is better than verification by replication whenever $N \cdot n < k$.

\section{Probabilistic analysis}
Let $X$ be a random variable outputting the number of nodes who voted that the computation was carried out correctly.
Let's consider two scenarios. 
\subsection{Bad provider trying to get away with an incorrect computation}
With sufficiently large number of verifying nodes $N$, we can expect approximately $pN$ honest and $(1-p)N$ dishonest nodes. 
Assume the worst case: all dishonest peers are colluding. This means that 
\begin{equation}
\frac{\mathbb{E}X}{N} = (1-p) + pc^n
\end{equation}
with $\lim_{n\to\infty} c^n = 0$ it's just the matter of adjusting the parameter $n$ so that the dishonest peers cannot overthrow the verdict being in minority.

\subsection{Good provider penalized by colluding dishonest peers}
In this case all $pN$ peers provide the truthful verdict, so the requirement $p > \frac{1}{2}$ is enough.

\section{Game-theoretical analysis} 
Peers need to have a financial incentive to participate in the verification process.
This can be solved very easily, by paying the participating peer a reward proportional
to the fraction of the subtask computed, i.e. $\frac{n}{k} P$, where $P$ is the price
to be earned for computing the whole task. Only peers whose result is consistent with
the majority vote (i.e. believed to be true) receive this reward.

This imposes the risk of peers colluding, whose alleviation is not much more difficult,
though. Instead of rewarding only the peers with a fixed amount of GNT, we set a pool
of $\frac{N \cdot n}{k} P$, distributed equally among the peers which are believed to have
calculated correctly. If $p = 0$, i.e. all peers are honest, this is equivalent to 
the previous reward system. 

The new system has this interesting property that it discourages collusion on a
game-theoretic background, just as proposed in the solution for verification by
replication. Precisely, if two peers decide to cooperate, compute the result once
(by a single party) and share the result with the other, the computing party's 
winning strategy is to properly recompute the chunk and give a false result to 
the other one. 

TODO what is the expected number of shared subtasks?
TODO why is it a winning strategy to compute everything?

\section{Cheap verdict collection}
The simple \textit{put it on blockchain} method is very inefficient when it 
comes to the Ethereum transaction fees. The number of blockchain writes
can be easily minimized using the following protocol:

The gathering of results consists of two phases, as previously. Instead of 
everyone putting his verdict on his own, a random node collects the results 
from the whole network (this can be a problem with a very large number, where
several nodes could be used instead).

All verdicts are, as previously encrypted and signed (in this order!). This
means that the verdict of any node cannot be tampered with - if it's present
in the report, it's exactly how the node voted. Thus, the only attack that
can be carried out in this model is not to include the verdicts from a
particular node in the voting summary, which is more likely in the second
phase, when the collector can decrypt the votes prior to submitting them 
onto the blockchain and omit some of them to malform the voting results.

This can be alleviated by allowing an \textit{objection} on the blockchain.
Precisely, after a node puts the summary on the blockchain, there is a 
timeout until these results become valid. Within this timeout any node can
raise an objection and announce that he is the next one to gather the results
from any node possibly ignored by the previous collectors.

This way, every node will eventually submit his results onto the blockchain.

TODO expected value of number of blockchain writes as a function of 
$p, N$
\section{Possible attacks}
\subsection{Sybil attack}
 
\section{Presentation on concrete parameter values}
TODO

\end{document}
